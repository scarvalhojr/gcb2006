\documentclass[english]{lni}

\IfFileExists{latin1.sty}{\usepackage{latin1}}{\usepackage{isolatin1}}

\usepackage{graphicx}

\newcommand{\ignore}[1]{}

\begin{document}

\author{S\'ergio A. de Carvalho Jr.\,$^{\rm a,b,c}$ and Sven Rahmann\,$^{\rm b,c}$}
\title{Microarray Layout as a Quadratic Assignment Problem}

%% no address command?
%% \address{$^{\rm a}$Graduiertenkolleg Bioinformatik, Bielefeld University, Germany,\\
%% $^{\rm b}$International NRW Graduate School in Bioinformatics and Genome Research,
%% Bielefeld University, Germany,\\
%% $^{\rm c}$Algorithms and Statistics for Systems Biology group, Genome Informatics,
%% Technische Fakult\"at, Bielefeld University, D-33594 Bielefeld, Germany.}

\maketitle

% ==============================================================================
\begin{abstract}
% ==============================================================================

The production of commercial DNA microarrays is based on a
light-directed chemical synthesis driven by a set of masks or
micromirror arrays. Because of the natural properties of light and the
ever shrinking feature sizes, the arrangement of the probes on the
chip and the order in which their nucleotides are synthesized play an
important role on the quality of the final product.
We propose a new model called \emph{conflict index} for evaluating
microarray layouts, and we show that the probe placement problem is an
instance of the \emph{quadratic assignment problem} (QAP), which opens
up the way for using QAP heuristics. We use a existing heuristic
called GRASP to design the layout of small artificial chips with
promising results. We compare this approach with the best known
algorithm and describe how it can be combined with other existing
algorithms to design the latest million-probe microarrays.

\end{abstract}

% ==============================================================================
\section{Introduction}
% ==============================================================================

An oligonucleotide microarray is a piece of glass or plastic on which
single-stranded fragments of DNA, called \emph{probes}, are affixed or
synthesized. The chips produced by Affymetrix, for instance, can contain more
than one million spots (or \emph{features}) as small as 11 $\mu$m, with each
spot accommodating several million copies of a probe. Probes are typically 25
nucleotides long and are synthesized in parallel, on the chip, in a series of
repetitive steps. Each step appends the same nucleotide to probes of selected
regions of the chip. Selection occurs by exposure to light with the help of a
photolithographic mask that allows or obstructs the passage of light
accordingly \cite{FODOR91}.

Formally, we have a set of probes $\mathcal{P} = \{p_{1}, p_{2}, ... p_{n}\}$
that are produced by a series of masks
$\mathcal{M} = (m_{1}, m_{2}, ... m_{\mu})$, where each mask $m_{k}$ induces the
addition of a particular nucleotide $t_{k} \in \{A, C, G, T\}$ to a subset
of~$\mathcal{P}$. The \emph{nucleotide deposition sequence}
$\mathcal{S} = t_{1} t_{2} \ldots t_{\mu}$ corresponding to the sequence of
nucleotides added at each masking step is therefore a supersequence of all
$p_{i} \in \mathcal{P}$ \cite{RAHMANN03}.

In general, a probe can be \emph{embedded} within $\mathcal{S}$ in several ways.
An embedding of $p_{i}$ is a $\mu$-tuple
$\varepsilon_{i} = (e_{i,1}, e_{i,2}, ... e_{i,\mu})$ in which $e_{i,k} = 1$ if
probe $p_{i}$ receives nucleotide $t_{k}$ (at step~$k$), or 0 otherwise
(Figure~\ref{fig:masking_process}).

\begin{figure}
\centerline{\includegraphics[width=230pt]{chip}}
\caption{Synthesis of a hypothetical 3\,x\,3 chip. On the top left, the chip
layout and its 3-base-long probe sequences. On the top right, the deposition
sequence and the probe embeddings. On the bottom, the first four resulting
photolithographic masks.}
\label{fig:masking_process}
\end{figure}

Deposition sequences are usually cyclical, that is $\mathcal{S}$ is a repeated
permutation of the alphabet. This is mainly because such sequences maximize the
number of possible subsequences \cite{CHASE76}. In this context, we can
distinguish between \emph{synchronous} and \emph{asynchronous} embeddings. In
the first case, each probe has one and only one nucleotide synthesized in every
cycle of the deposition sequence; hence, 100 masking steps are needed to
synchronously synthesize probes of length 25. In the case of asynchronous
embeddings, probes can have any number of nucleotides synthesized in any given
cycle. This allows for shorter deposition sequences. All Affymetrix chips that
we know of can be asynchronously synthesized in 74 masking steps\footnote{We
understand that Affymetrix uses the same truncated repetition of TGCA to
synthesize all of their chips, which suggests that their probe selection only
chooses probes that fit into that deposition sequence.}.

Because of diffraction of light or internal reflection, untargeted spots can
sometimes be accidentally activated in a certain masking step, producing
unpredicted probes that can compromise the results of an experiment. This issue
was described by \cite{FODOR91}, who noted that the problem is more likely to
occur near the borders between masked and unmasked spots. This observation has
given rise to the term \emph{border conflict}.

We are interested in finding an arrangement of the probes on the chip together
with their embeddings in such a way that we minimize the chances of unintended
illumination during mask exposure steps. This problem appears to be hard
(although the authors are not aware of an NP-hardness proof, and our QAP
formulation has several special properties) because of the exponential number
of possible arrangements, and optimal solutions are unlikely to be found even
for very small chips and even if we consider the probes as having a single
pre-defined embedding.

If we consider all valid embeddings, the problem is even harder. A typical
probe of an Affymetrix chip, for instance, can have up to several million
possible embeddings. For this reason, the problem has been traditionally
tackled in two phases. First, an initial embedding of the probes is fixed
and an arrangement of these embeddings on the chip with minimum border
conflicts is sought. This is usually referred to as the \emph{placement}
problem. Second, a \emph{post-placement} optimization phase re-embeds the probes
considering its location on the chip, in such a way that the conflicts with the
neighboring spots are further reduced.

In the next section, we review the Border Length Minimization Problem
\cite{HANNENHALLI02}, and define an extended model for evaluating microarray
layouts. In Section~\ref{sec:previous_work}, we briefly review existing
placement strategies. In Section~\ref{sec:qap}, we propose a new approach to
the design of microarrays based on the quadratic assignment problem (QAP). The
results of using a QAP heuristic algorithm, called GRASP, to design small
artificial chips are presented in Section \ref{sec:results}, where we also
compare its performance with the best known placement algorithm. Finally, we
discuss how this approach can be used to design larger microarrays
(Section~\ref{sec:discussion}).

% ==============================================================================
\section{Modeling}
% ==============================================================================
\label{sec:model}

\subsection{Border Length}

\cite{HANNENHALLI02} were the first to give a formal definition to the problem
of unintended illumination in the production of microarrays. They formulated the
\emph{Border Length Minimization Problem}, which aims at finding an arrangement
of the probes together with their embeddings in such a way the number of border
conflicts during mask exposure steps is minimal.

The \emph{border length}~$\mathcal{B}_k$ of a mask~$m_{k}$ is simply
defined as the number of borders shared by masked and unmasked spots
at masking step~$k$. The total border length of a given arrangement is
the sum of border lengths over all masks. For example, in
Figure~\ref{fig:masking_process}, the four masks shown have
$\mathcal{B}_1 = 4$, $\mathcal{B}_2 = 6$, $\mathcal{B}_3 = 6$ and
$\mathcal{B}_4 = 4$. The total border length of that arrangement is 50.

% ------------------------------------------------------------------------------
\subsection{Conflict Index}

The border length of an individual mask measures the quality of that
mask. We are more interested in estimating the risk of synthesizing a faulty
probe at a given spot. That is, we need a per-spot measure
instead of a per-mask measure. Additionally, \cite{KAHNG03A} noted
that the definition of border length does not take into account two
simple yet important practical considerations:
\begin{itemize}
\item[a)] stray light might activate not only adjacent neighbors but
  also probes that lie as far as three cells away from the targeted
  spot;
\item[b)] imperfections produced in the middle of a probe are more
  harmful than in its extremities.
\end{itemize}
This motivates the following definition of the \emph{conflict
  index}~$\mathcal{C}(s)$ of a spot~$s$ whose probe of
length~$\ell_{s}$ is synthesized in $\mu$~masking steps. First we
define a distance-dependent weighting function, $\delta(s,s',k)$, that
accounts for observation a) above:
%%
\begin{equation}
\label{eq:dist_weight}
\delta(s,s',k) :=
\left\{
	\begin{array}{ll}
		(d(s,s'))^{-2} & \mbox{if $s'$ is unmasked at step $k$}, \\
		0 & \mbox{otherwise}, \\
	\end{array}
\right.
\end{equation}
%%
where $d(s,s')$ is the Euclidean distance between spots~$s$ and~$s'$.
This form of weighting function is the same as suggested in
\cite{KAHNG03A}.  Note that $\delta$ is a ``closeness'' measure
between $s$ and $s'$ only if the spot in the neighboring spot $s'$ is
not masked (and thus creates the potential of illumination at $s$). To
restrict the number of neighbors that need to be considered, we
restrict the support of $\delta(s,s',\cdot)$ to those $s'\neq s$ that
are in a $7\times 7$ grid centered around $s$ (see
Figure~\ref{fig:conflictindex}~top).


We use position-dependent weights to account for observation b):
%%
\begin{equation}\label{eq:pos_mult}
\omega(s,k) :=
\left\{
	\begin{array}{ll}
		c \cdot \exp{\left(\theta \cdot \lambda(s,k)\right)} & \mbox{if $s$ is masked at step $k$}, \\
		0 & \mbox{otherwise}, \\
	\end{array}
\right.
\end{equation}
%%
where $c>0$ and $\theta>0$ are constants, and
%%
\begin{equation}\label{eq:base_pos}
  \lambda(s,k) := 1 + \min(b_{s,k},\ell_{s} - b_{s,k})
\end{equation}
%%
is the distance of the nucleotide synthesized in step $k$ (if any)
from the start or end of the probe: $b_{s,k}$ denotes the number of
nucleotides synthesized at spot~$s$ up to and including step~$k$ and
$\ell_s$ is the probe length (see Figure~\ref{fig:conflictindex}
bottom). We set the constants $c$ and $\theta$ as follows:
\[ \theta = \frac{5}{\ell_s}; \qquad c = \frac{1}{\exp{\theta}}. \]
It is generally agreed that the chances of a successful hybridization
between probe and target are higher if a mismatched base occurs at the
extremities of the formed duplex instead of at its center. The precise
effects of this position, however, is not yet fully understood and has
been an active topic of research \cite{BINDER05}. The motivation
behind an exponentially increasing weighting function is that the
probability of a successful stable hybridization of a probe with its
target should increase exponentially with the absolute value of its
Gibbs free energy, which increases linearly with the length of the
longest perfect match between probe and target. The parameter $\theta$
controls how steeply the exponential weighting function rises towards
the middle of the probe.

We now define the conflict index of a spot $s$ as
\begin{equation}
\label{eq:conf_idx}
\mathcal{C}(s) := \sum_{k=1}^{\mu} \left( \omega(s,k) \sum_{s'} \delta(s,s',k) \right),
\end{equation}
%%
where $s'$ ranges over all spots that are at most three cells away
from $s$.  $\mathcal{C}(s)$ can be interpreted as the fraction of
faulty probes (because of unwanted illumination) produced at spot $s$.

\begin{figure}
%%
\footnotesize{ \centerline{
\begin{tabular}{c|c|c|c|c|c|c|c|} \cline{2-8}
   & 0.06 & 0.08 & 0.10 & 0.11 & 0.10 & 0.08 & 0.06 \\ \cline{2-8}
   & 0.08 & 0.13 & 0.20 & 0.25 & 0.20 & 0.13 & 0.08 \\ \cline{2-8}
   & 0.10 & 0.20 & 0.50 & 1.00 & 0.50 & 0.20 & 0.10 \\ \cline{2-8}
   & 0.11 & 0.25 & 1.00 &  s   & 1.00 & 0.25 & 0.11 \\ \cline{2-8}
   & 0.10 & 0.20 & 0.50 & 1.00 & 0.50 & 0.20 & 0.10 \\ \cline{2-8}
   & 0.08 & 0.13 & 0.20 & 0.25 & 0.20 & 0.13 & 0.08 \\ \cline{2-8}
   & 0.06 & 0.08 & 0.10 & 0.11 & 0.10 & 0.08 & 0.06 \\ \cline{2-8}
\end{tabular}
}}
%%
\footnotesize{\centerline{
%%
\begin{picture}(415,165)
%% \put(0,123){a)}
%% \put(0,41){b)}
\put(15,0){\makebox(395,165){
%GNUPLOT: LaTeX picture with Postscript
\begin{picture}(0,0)%
\includegraphics{position_weights}%
\end{picture}%
\begingroup
\setlength{\unitlength}{0.0200bp}%
\begin{picture}(9000,5400)(0,0)%
\put(1250,1500){\makebox(0,0)[r]{\strut{} 0}}%
\put(1250,2067){\makebox(0,0)[r]{\strut{} 2}}%
\put(1250,2633){\makebox(0,0)[r]{\strut{} 4}}%
\put(1250,3200){\makebox(0,0)[r]{\strut{} 6}}%
\put(1250,3767){\makebox(0,0)[r]{\strut{} 8}}%
\put(1250,4333){\makebox(0,0)[r]{\strut{} 10}}%
\put(1250,4900){\makebox(0,0)[r]{\strut{} 12}}%
\put(1630,1000){\makebox(0,0){\strut{} 0}}%
\put(2928,1000){\makebox(0,0){\strut{} 5}}%
\put(4226,1000){\makebox(0,0){\strut{} 10}}%
\put(5524,1000){\makebox(0,0){\strut{} 15}}%
\put(6822,1000){\makebox(0,0){\strut{} 20}}%
\put(8120,1000){\makebox(0,0){\strut{} 25}}%
%\put(4875,250){\makebox(0,0){\strut{}$b_{p,t}$}}%
\end{picture}%
\endgroup
\endinput

}}
% \includegraphics*[0mm,0mm][128mm,65mm]{division}
\end{picture}\vspace*{-3ex}
%%
}}
%%
\caption{Ranges of values for both $\delta$ and $\omega$ on a typical Affymetrix
chip where probes of length~$\ell = 25$ are synthesized in $\mu = 74$ masking
steps. Top: Distance-dependent weighting function $\delta(s,s',k)$ for a
spot~$s$ (shown in the center) and all close neighbors $s'$, assuming that $s'$
is unmasked at step $k$.  Bottom: Position-dependent weights $\omega(s,k)$ at
each value of $b_{s,k}$, assuming that spot $s$ is masked at step $k$.}
\label{fig:conflictindex}
\end{figure}

Finally, we note the following relation between conflict indices and border
lengths: Define $\delta(s,s',k):=1$ if $s'$ is a direct neighbor of $s$ and is
unmasked in step $k$, and $:=0$ otherwise.  Also define $\omega(s,k):=1$ if
$s$ is masked in step $k$, and $:=0$ otherwise. Then $\sum_s\, \mathcal{C}(s)
= 2 \sum_{k=1}^\mu\, \mathcal{B}_k$, as each border conflict is counted twice:
for $s'$ and $s$.

This shows that total border length and total conflict are correlated: A good
layout is one with low border length as well as low average conflict index,
although it is clearly possible to observe a reduction of the conflict index
at the expense of an increase in border length, and vice-versa.


% ==============================================================================
\section{Previous Work}
% ==============================================================================
\label{sec:previous_work}

We briefly review existing layout techniques, distinguishing between placement
algorithms and partitioning algorithms. Post-placement optimizations such as the
Chessboard \cite{KAHNG02} are not covered.

% ------------------------------------------------------------------------------
\subsection{Placement Algorithms}
\label{sec:placement_alg}

\ignore{The first to formally address the border length problem were
\cite{FELDMAN93}. They showed how an optimal placement can be constructed based
on a two-dimensional Gray code. However, their work is restricted to
\emph{uniform arrays} (arrays containing all possible probes of a given length)
and synchronous embeddings.}

\cite{HANNENHALLI02} were the first to consider the border length problem on
large oligonucleotide arrays of arbitrary probes. They reported that the first
Affymetrix chips were designed using a heuristic for the traveling salesman
problem (TSP). The idea consists of building a weighted graph with nodes
representing probes and edges containing the Hamming distance between the
probes. A TSP tour is approximated, resulting in consecutive probes in the tour
being likely to be similar. The TSP tour is finally \emph{threaded} on the array
in a row-by-row fashion. \cite{HANNENHALLI02} suggested a different threading of
the TSP tour on the chip, called \emph{1-threading}, to achieve up to 20\%
reduction in border length.

\cite{KAHNG02} propose the \emph{epitaxial placement algorithm} that
places a random probe in the center of the array and continues to
insert probes in spots adjacent to already filled spots, employing a
greedy heuristic to select the next spot to be filled and the probe
that is assigned to it. Priority is given to spots whose
neighbors are already filled, in which case the algorithm places the
probe with minimum sum of Hamming distances to its neighbors. If no
such a spot exists, the algorithm examines all non-filled spots~$s_i$
with $n_i \geq 1$ filled neighbors and finds a non-assigned probe
$p_j$ with minimum sum of Hamming distances to the neighboring probes
$H_{ij}$. For each possible assignment of $p_j$ to $s_i$, it computes
a cost $c(s_i,p_j) := k_{n_i} H_{ij} / n$, where $k_{n_i}$ are scaling
coefficients ($k_1 = 1$, $k_2 = 0.8$, and $k_3 = 0.6$), and makes the
assignment with minimum cost. With this algorithm, they claimed to
achieve up to 10\% reduction in border conflicts over the TSP-based
approach of \cite{HANNENHALLI02}.

\ignore{
The major problem with the epitaxial and the TSP-based algorithm is that they
have at least quadratic time complexity and thus are not scalable for the
latest million-probe microarrays. According to their experiments, the TSP
approach needed around 32 minutes to produce the layout of a 200\,x\,200 chip,
whereas the epitaxial algorithm needed 74 minutes on average. For a 500\,x\,500
chip, the TSP took over 30 hours to complete, whereas the epitaxial algorithm
did not complete ``due to prohibitively large running time or memory
requirements'' \cite{KAHNG02}.
}

\ignore{
This observation has led to the development of two new algorithms by
\cite{KAHNG03A}. The first one, called sliding-window matching (SWM), is not
exactly a placement algorithm as it iteratively improves an initial placement
that can be constructed by, for instance, TSP and 1-threading. Improvements
are achieved by selecting an independent set of spots inside the window and
optimally replacing their probes using a minimum-weight perfect matching
algorithm. The term independent refers to probes that can be replaced without
affecting the border length of the other selected probes.
}

Both the TSP and the epitaxial approach do not scale well to large
chips.  Another algorithm described by \cite{KAHNG03A} is a variant
of the epitaxial algorithm, called \emph{row-epitaxial}, with two main
differences: spots are filled in a pre-defined order, namely
row-by-row, and only probes of a limited list of candidates $Q$ are
considered when filling each spot. Their experimental results showed
that the row-epitaxial is the best large-scale placement algorithm,
achieving up to 9\% reduction in border length when compared to the
TSP-based approach of \cite{HANNENHALLI02}.

% ------------------------------------------------------------------------------
\subsection{Partitioning Algorithms}
\label{sec:partition}

The ever growing number of probes on the latest microarrays and the properties
of the placement problem naturally suggest the use of partitioning strategies to
reduce the running time of the algorithms. The placement problem can be
partitioned by dividing the set of probes into smaller sub-sets, and assigning
these sub-sets to sub-regions of the chip. Each sub-region can then be treated
as an independent chip or recursively partitioned. These smaller sub-problems,
when solved, immediately constitute a final solution. In this way, algorithms
with non-linear time or space complexities can be used compute the layout of
larger chips that otherwise would not be feasible. A partitioning is clearly a
compromise in solution quality. However, due to the large number of probes, this
compromise can be small, specially if the partitioning is able to place similar
probes together.

The only partitioning algorithm available in the literature is the
centroid-based quadrisection \cite{KAHNG03B}. It is a recursive procedure that
works as follows. First, it randomly selects a probe $c_1$ from the probe set
$\mathcal{P}$. Then, it examines all other probes of $\mathcal{P}$ and selects
$c_2$ with maximum $h(c_1,c_2)$, where $h(c_1,c_2)$ is the Hamming distance
between the embeddings of $c_1$ and $c_2$. Similarly it finds $c_3$ with maximum
$h(c_1,c_3) + h(c_2,c_3)$ and $c_4$ with maximum
$h(c_1,c_4) + h(c_2,c_4) + h(c_3,c_4)$. Probes $c_1$, $c_2$, $c_3$ and $c_4$ are
called centroids. All other probes $p_i \in \mathcal{P}$ are then compared to
the centroids and assigned to the sub-set $\mathcal{P}_j$ associated with $c_j$
with minimum $h(p_i,c_j)$. Each sub-set $\mathcal{P}_j$ is assigned to a
sub-region of the chip. The procedure is repeated recursively on each sub-region
until a given recursion depth is reached.

The result of this algorithm is a partitioning of the chip into several
sub-regions and an assignment of sub-sets of $\mathcal{P}$ to each sub-region.
For the actual placement of the probes in each sub-region, another algorithm is
needed. For this purpose, \cite{KAHNG03B} have used the row-epitaxial algorithm.

\ignore{
Their results show that the running time of the row-epitaxial algorithm
drops significantly with increasing recursion depth. The time required to place
the probes of a 500\,x\,500 chip, for instance, dropped by 69\% with $L = 3$
when compared with the time required by the row-epitaxial without any
partitioning.
It is not clear from their experiments, however, how the choice of $L$ impaired
the performance of the row-epitaxial algorithm in terms of solution quality
since they have restricted their experiments to $L \leq 3$. Moreover, there is
no clear trend toward reduction or increase in border length as $L$ varies
from~0 to~3.
}

% ==============================================================================
\section{Quadratic Assignment Problem}
% ==============================================================================
\label{sec:qap}

We now explore a different approach to the design of microarrays based on the
quadratic assignment problem (QAP), a classical combinatorial optimization
problem introduced by \cite{KOOPMANS57}.

The QAP can be stated as follows. Given $n \times n$ real-valued matrices $F =
(f_{ij})\geq 0$ and $D = (d_{ij})\geq 0$, find a permutation $\pi$ of $\{1, 2,
\ldots n\}$ such that
\begin{equation}\label{eq:qap_def}
  \sum_{i=1}^{n} \sum_{j=1}^{n}\,  f_{ij} \cdot d_{\pi(i)\pi(j)} \to \min.
\end{equation}

The attribute \emph{quadratic} stems from the fact that the target function
can be written with $n^2$ binary indicator variables $x_{ik}\in\{0,1\}$, where
$x_{ik}:=1$ if and only if $k=\pi(i)$. The objective~(\ref{eq:qap_def}) then
becomes
\begin{equation}\label{eq:qap_x}
  \sum_{i=1}^{n} \sum_{j=1}^{n}\,  f_{ij} \cdot 
  \sum_{k=1}^{n} \sum_{l=1}^{n}\,  d_{kl} \cdot x_{ik}\cdot x_{jl}
  \to \min
\end{equation}
such that $\sum_{k}\, x_{ik}=1$ for all $i$, $\sum_{i}\, x_{ik}=1$ for all $k$
and $x_{ik}\in\{0,1\}$ for all $(i,k)$. The objective function is a quadratic
form in $x$.

The QAP has been used to model a variety of real-life problems. One of its
major applications is to model the facility location problem where $n$
facilities must be assigned to $n$ locations. In this scenario, $F$ is called
the flow matrix as $f_{ij}$ represents the flow of materials from facility $i$
to facility $j$. One unit of flow is assumed to have an associated cost
proportional to the distance between the facilities. Matrix $D$ is called the
distance matrix as $d_{kl}$ gives the distance between locations $k$ and $l$.
The optimal permutation $\pi$ defines a one-to-one assignment of facilities to
locations with minimum cost.

% ------------------------------------------------------------------------------
\subsection{QAP Formulation of Probe Placement}
\label{sec:qap_form}

The probe placement problem can be seen as an instance of the QAP. To
formulate it, we use the facility location example by viewing the probes as
locations and the spots as facilities, i.e., the spots are assigned to the
probes. In this case the flow matrix $F$ contains the ``closeness'' values
between spots, and the distance matrix $D$ contains the (weighted) Hamming
distances between probe embeddings.  We first give the general formulation for
conflict index; the case of border length minimization is obtained by using
the particular weight functions given at the end of
Section~\ref{sec:model}.

Obviously, the QAP requires a one-to-one correspondence between spots and
probes. In a realistic setting, we may have more spots available than probes
to place. This does not cause problems: Below we show that we simply need to
add enough ``empty'' probes and define their weight functions appropriately.

Perhaps more severely, we assume that all probes have a single pre-defined
embedding in order to force a one-to-one relationship.  A more elaborate
formulation would consider all possible embeddings of a probe, but then it
becomes necessary to ensure that only one embedding of a probe is assigned to
a spot. This still leads to a quadratic integer programming problem, albeit
with slightly different side conditions than for~(\ref{eq:qap_x}).

Our goal is to design a microarray minimizing the sum of conflict indices over
all spots $i$, i.e., $\sum_{i} \mathcal{C}(i) \to \min$.


\paragraph{Defining $f_{ij}$.}
The ``flow'' between spots $i$ and $j$ depends on their Euclidean distance
$d(i,j)$ on the array; in accordance with the conflict index model, we set
\begin{equation}
  f_{ij} := \left\{ \begin{array}{ll}
      (d(i,j))^{-2} & \mbox{if spot $j$ is ``near'' spot $i$}, \\
      0 & \mbox{otherwise}. \\
    \end{array} \right.
\end{equation}
%%
where ``near'' means that spot~$j$ is at most three cells away from~$i$. Note
that most of the flow values on large arrays are zero. For Border Length
Minimization, the case is even simpler: We set $f_{ij}:=1$ if spots $i$ and
$j$ are direct neighbors, and $f_{ij}:=0$ otherwise.

\paragraph{Defining $d_{kl}$.}
The ``distance'' between probes $k$ and $l$ depends on the (weighted) Hamming
distance of their embeddings. To account for possible ``empty'' probes to fill
up surplus spots, we set $d_{kl}:=0$ if $k$ or $l$ or both refer to an empty
probe (i.e., empty probes never contribute to the target function, we do not
mind if nucleotides are erroneously synthesized on spots assigned to empty
probes). 

For real probes, we set
\[ d_{kl} := \sum_{t=1}^\mu\, d_{klt}, \]
where we now use $t$ as the synthesis step index (the $k$ used in
Section~\ref{sec:model} is now a probe index). The number $d_{klt}$ is the
potential contribution of probe $l$'s embedding to the failure risk of probe
$k$ in the $t$-th synthesis step. According to the conflict index model,
\[ d_{klt}  = \left\{ \begin{array}{ll}
    c \cdot \exp(\theta \cdot \lambda'(k,t)) 
    & \mbox{if $k$ masked, $l$ unmasked}\\
    & \mbox{in step $t$,}\\
    0
    & \mbox{otherwise.}
  \end{array} \right.
\]
Because of the change in notation and reference to probes instead of spots, we
repeat that $\lambda'(k,t) = 1 + \min(b'_{k,t},\ell'_{k} - b'_{k,t})$, where
$\ell'_{k}$ is the length of $p_k$ and $b'_{k,t}$ is the number of nucleotides
of $p_k$ synthesized up to and including step $t$.

In the special case of the Border Length Minimization Problem, where
$\theta=0$ and $c=1/2$, we obtain that $d_{kl} + d_{lk} = H_{kl} = H_{lk}$,
where $H_{kl}$ denotes the  Hamming distance between the embeddings of probes
$k$ and $l$.


\paragraph{Proof of correctness.}
It now follows that for a given assignment (permutation) $\pi$, we have
$f_{ij} d_{\pi(i)\pi(j)} = \sum_{t=1}^\mu\, \delta(i,j,t) \cdot \omega(i,t)$
in the notation of Section~\ref{sec:model}. Therefore the objective
function~(\ref{eq:qap_def}) becomes
\begin{eqnarray*}
  & & \sum_i \sum_j\, f_{ij} \cdot d_{\pi(i)\pi(j)}\\
  &=& \sum_i \sum_j\, \left( \sum_{t=1}^\mu\, \delta(i,j,t) \cdot \omega(i,t)  \right)\\
  &=& \sum_i \sum_{t=1}^\mu\, \left( \omega(i,t) \cdot \sum_j\, \delta(i,j,t)  \right)\\
  &=& \sum_i \mathcal{C}_i,
\end{eqnarray*}
and indeed equals the total conflict index with our definitions of $(f_{ij})$
and $(d_{kl})$.


\paragraph{Remark.} 
Note that it is technically possible to switch the definitions of $F=(f_{ij})$
and $D=(d_{kl})$, i.e., to assign probes to spots, without modifying the
problem formulation, but this would lead to high distance value for
neighboring spots many zero distance values for independent spots, a somewhat
counterintuitive model. Also, QAP heuristics tend to find pairs objects with
large flow values and place them close to each other initially. Therefore, the
way of modeling $F$ and $D$ is significant.


% ------------------------------------------------------------------------------
\subsection{QAP Heuristics}

In the previous sub-section we showed how the microarray placement problem can
be modeled as a quadratic assignment problem. This is interesting because we can
now use existing QAP algorithms to design the layout of microarrays minimizing
either the sum of border lengths or conflict indices. 

The QAP is known to be NP-hard and NP-hard to approximate. Instances of size
larger than $n = 20$ are generally considered to be impossible to solve (to
optimality). Fortunately, several heuristics are available including
approaches based on tabu search, simulated annealing and genetic algorithms
(see \cite{CELA98}, for a survey).

We now briefly describe GRASP, a heuristic proposed by \cite{FEO95}, which
was first used for solving the QAP by \cite{LI94}. We also outline a GRASP
variant known as GRASP with path-relinking \cite{OLIVEIRA04} that we have
used in the design of microarray chips. In the description that follows we use
the terms of the facility location problem: $f_{ij}$ is the flow between
facilities $i$ and $j$, $d_{kl}$ is the distance between locations $k$ and
$l$.

GRASP is an acronym for Greedy Randomized Adaptive Search Procedure. It is a
process comprised of two phases: a construction phase where it builds a random
feasible solution, and a local search phase where it seeks a local optimum in
the neighborhood of that solution. GRASP performs a number of independent
iterations of these two phases and outputs the best solution found.

Before the first iteration, GRASP sorts the $(n^2 - n)$ elements of the
distance matrix in increasing order, keeping the first $N:= \lfloor \beta (n^2 -
n) \rfloor$, where $0 < \beta < 1$ is a restriction parameter.
%%
\begin{displaymath}
d_{k_1 l_1} \le d_{k_2 l_2} \le \cdots \le d_{k_N l_N},
\end{displaymath}
%%

It also sorts the $(n^2 - n)$ elements of the flow matrix in decreasing order,
again keeping the first $N$:
%%
\begin{displaymath}
f_{i_1 j_1} \ge f_{i_2 j_2} \ge \cdots \ge f_{i_N j_N}.
\end{displaymath}

Finally, it sorts the costs of assigning initial pairs of facilities to pairs
of locations: The cost of initially assigning facility $i_q$ to location $k_q$
and facility $j_q$ to location $l_q$
for some $q\in\{1,\ldots,N\}$ is $d_{k_q l_q} f_{i_q j_q}$. GRASP sorts
the vector
%%
\begin{displaymath}
(d_{k_1 l_1}  f_{i_1 j_1},\;
d_{k_2 l_2}  f_{i_2 j_2},\; \ldots,\;
d_{k_N, l_N}  f_{i_N j_N}),
\end{displaymath}
%%
keeping the $\lfloor \alpha N \rfloor$ smallest elements, where $0 < \alpha <
1$ is another restriction parameter.

The construction phase of GRASP consists of two stages. In the first stage,
the algorithm makes a simultaneous assignment selected at random among those
with the $\lfloor \alpha \beta (n^2 - n) \rfloor$ smallest costs.

In the second stage of the construction phase, GRASP builds a feasible
solution by making a series of greedy assignments as follows. First, it
computes the costs of all $m$ remaining possible assignments with respect to
assignments already made. Then, it randomly selects one assignment among those
with $\lfloor \alpha m \rfloor$ smallest costs.

In the local search phase, GRASP searches for a local optimum in the
neighborhood of the constructed solution. Different search strategies and
different definitions of the neighborhood can be used. One possible approach is
to check every possible swap of assignments and make those which improve the
current solution until no further improvements can be made.

GRASP repeats the construction and local search phases for a given number of
times, keeping the best solution found. Each iteration is independent in the
sense that every construction phase builds a new solution from scratch. The best
solutions are kept, but GRASP takes no advantage of the knowledge gained in
previously iterations to build or improve a new solution. This is exactly where
the concept of path-relinking comes into play.

GRASP with path-relinking is an extension of the basic GRASP that uses an elite
set $P$ to store the best solutions found. It also incorporates a third phase
that consists of choosing at random one elite solution $q \in P$  to be combined
with the last solution $p$ produced after the local search phase.

Solutions $p$ and $q$ are combined as follows. For every location
$k = 1, \ldots, n$, the path-relinking algorithm attempts to exchange facility
$p_k$ assigned to location $k$ in  solution $p$ with facility $q_k$ assigned to
location $k$ in the elite solution. In order to keep the solution $p$ feasible,
it actually exchanges $p_k$ with $p_l$, where $p_l = q_k$. This exchange is
performed only if it results in a better solution. The result of the
path-relinking phase is a solution $r$ that is as good as $p$ and $q$.

For more details on GRASP with path-relinking, we refer to~\cite{OLIVEIRA04}.

% ==============================================================================
\section{Results}
% ==============================================================================
\label{sec:results}

We present experimental results of using GRASP with path-relinking (GRASP-PR)
for designing the layout of small artificial chips.

Because of the large number of probes on industrial microarrays, it is not
feasible to use GRASP-PR (or any other QAP method) to design the layout of an
entire microarray chip. However, it is certainly possible to use it on small
sub-regions of a chip. This is interesting because we can combine GRASP-PR
with a partitioning strategy such as the centroid-based quadrisection
described in Section \ref{sec:partition}.

We have run GRASP-PR as well as the best known placement algorithm,
row-epitaxial (see Section \ref{sec:placement_alg}), on several small random
chips. For GRASP-PR, we used a C implementation provided by
\cite{OLIVEIRA04}, with default parameters: 32 iterations, $\alpha=0.1$,
$\beta=0.5$, and elite set with size $|P|=10$. The main routine takes
three arguments: matrices $F$ and $D$ and the dimension of the problem~$n$ (in
our case, the number of spots or probes). We generated the matrices using the
formulations presented in Section \ref{sec:qap_form}. For the row-epitaxial
algorithm we used a C implementation provided by \cite{KAHNG03A}. The
running times and the border length of the resulting layouts are shown in
Table \ref{tab:graspr_reptx}.

\begin{table}[t]
\caption{Border length of random chips compared with the layouts produced by
row-epitaxial and GRASP with path-relinking. Reductions in border length are
reported in percentages compared to the initial layout. Chips contain 25-base
long probes uniformly generated and synchronously embedded. Border length and
running times are averages over a set of five chips.\label{tab:graspr_reptx}}
{\begin{tabular}{crrcrrrcrrr}
          &            & Random & & \multicolumn{3}{c}{Row-epitaxial}  & & \multicolumn{3}{c}{GRASP with path-relinking}  \\ \cline{3-3} \cline{5-7} \cline{9-11}
Chip      & Number of  & Border & & Border & Reduction & Time          & & Border & Reduction & Time   \\
dimension & probes     & length & & length & (\%)      & (sec.)        & & length & (\%)      & (sec.) \\

6\,x\,6   &  36 & 2\,239.20 & & 1\,942.40 & 13.25 & 0.010 & & 1\,882.40 & 15.93 & 2.991   \\
7\,x\,7   &  49 & 3\,115.20 & & 2\,675.60 & 14.11 & 0.020 & & 2\,621.60 & 15.84 & 7.074   \\
8\,x\,8   &  64 & 4\,202.40 & & 3\,514.00 & 16.38 & 0.024 & & 3\,481.20 & 17.16 & 13.568  \\
9\,x\,9   &  81 & 5\,420.00 & & 4\,471.20 & 17.51 & 0.028 & & 4\,460.40 & 17.70 & 28.076  \\
10\,x\,10 & 100 & 6\,740.40 & & 5\,556.20 & 17.57 & 0.034 & & 5\,536.00 & 17.87 & 55.430  \\
11\,x\,11 & 121 & 8\,212.00 & & 6\,726.80 & 18.09 & 0.040 & & 6\,734.80 & 17.99 & 84.659  \\
12\,x\,12 & 144 & 9\,872.00 & & 7\,975.20 & 19.21 & 0.044 & & 8\,038.00 & 18.58 & 148.196 \\
\end{tabular}}{Experiments were conducted on a Sun Fire V1280 server with 900Mhz
UltraSparc III+ processors and 96 Gb of RAM under similar load balances.}
\end{table}

Our results show that GRASP-PR produces layouts with lower border lengths than
the row-epitaxial algorithm for the smaller chips. For 6\,x\,6 chips GRASP-PR
outperforms row-epitaxial by $2.5$ percentage points on average when compared to
the initial random layout. This is a promising result given that GRASP-PR is a
general QAP heuristic. For 10\,x\,10 chips, however, this difference drops to
$0.6$ percentage point. The row-epitaxial generates better layouts for 11\,x\,11
or larger chips. This is probably because a larger chip means that there are
more probes to choose from when filling the spots.

In terms of running time, the row-epitaxial is faster and shows little variation
as the number of probes grows. In contrast, the time required to compute a
layout with GRASP-PR increases at a fast rate.

% ==============================================================================
\section{Discussion}
% ==============================================================================
\label{sec:discussion}

We have identified the probe placement or microarray layout problem with
general distance-dependent and position-dependent weights as a (specially
structured) quadratic assignment problem. QAPs are notoriously hard to solve,
and currently known exact methods start to take prohibitively long already for
slightly more than $20$ objects, i.e., we barely could solve the problem for
$5\times 5$ arrays. However, the literature on QAP heuristics is quite rich,
as many problems in operations research can be modeled as QAPs. Here we used
one such heuristic to identify the potential of the
probe-placement-QAP-relation.

It is interesting to extrapolate the times shown on Table~\ref{tab:graspr_reptx}
to predict the total time that would be required to design the layout of
commercial microarrays, if we were to combine GRASP-PR with a partitioning
algorithm. If the partitioning produced 6\,x\,6 regions, 37\,636 sub-regions
would be created from the 1164\,x\,1164 Affymetrix Human Genome U133 Plus 2.0
GeneChip\raisebox{.6ex}{\scriptsize \textregistered}, one of the largest
Affymetrix chips. Since each sub-region takes around 3 seconds to compute with
GRASP-PR, the total time required for designing such a chip would be a little
over 31 hours (ignoring the time for the partitioning itself).

If the partitioning produced 12\,x\,12 regions, 9\,409 sub-regions would be
created and, at 2.4 minutes each, the total time would be more than 16 days.
This is probably prohibitive, although it is certainly possible to reduce the
time of each GRASP-PR execution by running it on faster machines or using a
parallel implementation (GRASP is known to be easily parallelized; see
\cite{LI94}). Figure~\ref{fig:time_extrapolation} shows similar predictions
based on our results with varying chip dimensions and partitioning sizes.

We believe that solution quality is more important than the running time of a
placement algorithm. Even if an algorithm takes a couple of days to complete,
it is time well spent given that commercial microarrays are likely to be
produced in large quantities. This is specially true when we consider the time
required for the whole design process of a microarray chip. Even customer
designed chips, that usually have a limited number of produced units, are likely
to benefit from a few extra hours of computing time.

\paragraph{Future Work.}
As mentioned earlier, partitioning is a compromise in solution quality in favor
of running time. However, it is not clear yet how the choice of the maximum
recursion depth in the centroid-based quadrisection can undermine the
effectiveness of the placement algorithms. At the moment, we are investigating
alternative partitioning strategies and evaluating their effects on the quality
of the solutions.

We conclude by noting that QAP heuristics such as GRASP-PR could also be used to
improve existing layouts if small changes were introduced. The idea is that we
can run such algorithms iteratively in a sliding-window fashion, where each
iteration produces an instance of a QAP whose size equals the size of the
window. The QAP heuristic can then be executed to check whether a different
arrangement of the probes inside the window can reduce the conflicts.

The only problem with this approach is that the QAP heuristic also needs to take
into account the conflicts due to the spots around the window. Otherwise, the
new layout may increase the conflicts on the borders of the window.

A possible solution to this problem is to solve a larger QAP instance
consisting of the spots inside the window as well as those around it. The
spots outside the window obviously must remain unchanged, and that can be done
by fixing the corresponding elements of the permutation $\pi$. Note also that
there is no need to compute $f_{ij}$ if spots $i$ and $j$ are both outside the
window, nor $d_{kl}$ if probes $k$ and $l$ are assigned to spots outside the
window.

\begin{figure}
{\footnotesize \centerline{%GNUPLOT: LaTeX picture with Postscript
\begin{picture}(0,0)%
\includegraphics{time_extrapolation}%
\end{picture}%
\begingroup
\setlength{\unitlength}{0.0200bp}%
\begin{picture}(12599,9180)(0,0)%
\put(2000,1500){\makebox(0,0)[r]{\strut{} 0}}%
\put(2000,2397){\makebox(0,0)[r]{\strut{} 50}}%
\put(2000,3295){\makebox(0,0)[r]{\strut{} 100}}%
\put(2000,4192){\makebox(0,0)[r]{\strut{} 150}}%
\put(2000,5090){\makebox(0,0)[r]{\strut{} 200}}%
\put(2000,5987){\makebox(0,0)[r]{\strut{} 250}}%
\put(2000,6885){\makebox(0,0)[r]{\strut{} 300}}%
\put(2000,7782){\makebox(0,0)[r]{\strut{} 350}}%
\put(2000,8680){\makebox(0,0)[r]{\strut{} 400}}%
\put(11370,1000){\makebox(0,0){\strut{}12\,x\,12}}%
\put(9530,1000){\makebox(0,0){\strut{}11\,x\,11}}%
\put(7850,1000){\makebox(0,0){\strut{}10\,x\,10}}%
\put(6330,1000){\makebox(0,0){\strut{}9\,x\,9}}%
\put(4970,1000){\makebox(0,0){\strut{}8\,x\,8}}%
\put(3770,1000){\makebox(0,0){\strut{}7\,x\,7}}%
\put(2730,1000){\makebox(0,0){\strut{}6\,x\,6}}%
\put(500,5090){\rotatebox{90}{\makebox(0,0){\strut{}Hours}}}%
\put(7050,250){\makebox(0,0){\strut{}Maximum dimension of partitions}}%
\put(4300,8130){\makebox(0,0)[l]{\strut{}Human U133 Plus 2.0 (1164\,x\,1164)}}%
\put(4300,7630){\makebox(0,0)[l]{\strut{}Zebra Fish (712\,x\,712)}}%
\put(4300,7130){\makebox(0,0)[l]{\strut{}E.Coli 2.0 (478\,x\,478)}}%
\end{picture}%
\endgroup
\endinput
}}
\caption{Predicted running times required to design selected Affymetrix GeneChip
arrays using GRASP-PR and a partitioning algorithm with varying degrees of
partitioning (based on data shown in Table~\ref{tab:graspr_reptx}). The
dimensions of the chips are shown in parentheses. The time required for the
partitioning itself is ignored.}\label{fig:time_extrapolation}
\end{figure}

% ==============================================================================
\begin{thebibliography}{}
% ==============================================================================

\bibitem[Binder and Preibisch, 2005]{BINDER05} Binder,H. and Preibisch,S. (2005)
Specific and nonspecific hybridization of oligonucleotide probes on microarrays.
{\it Biophysical Journal}, {\bf 89}, 337--352.

\bibitem[\c{C}ela, 1998]{CELA98} \c{C}ela,E. (1998) {\it The Quadratic
Assignment Problem: Theory and Algorithms}. Kluwer, Massachessets, USA.

\bibitem[Chase, 1976]{CHASE76} Chase,P.J. (1976) Subsequence numbers and
logarithmic concavity. {\it Discrete Mathematics} {\bf 16}, 123--140.

\bibitem[Feldman and Pevzner, 1994]{FELDMAN93} Feldman,W. and Pevzner,P. (1994)
Gray code masks for sequencing by hibridization. {\it Genomics}, {\bf 23},
233--235.

\bibitem[Feo and Resende, 1995]{FEO95} Feo,T.A. and Resende,M.G.C. (1995) Greedy
randomized adaptive search procedures. {\it Journal of Global Optimization},
{\bf 6}, 109--133.

\bibitem[Fodor {\it et~al}., 1991]{FODOR91} Fodor,S., Read,J., Pirrung,M.,
Stryer,L., Lu,A. and Solas,D. (1991) Light-directed, spatially addressable
parallel chemical synthesis. {\it Science}, {\bf 251}, 767--73.

\bibitem[Hannenhalli {\it et~al}., 2002]{HANNENHALLI02} Hannenhalli,S.,
Hubell,E., Lipshutz,R. and Pevzner,P. (2002) Combinatorial algorithms for design
of DNA arrays. {\it Advances in Biochemical Engineering / Biotechnology},
{\bf 77}, 1--19.

\bibitem[Kahng {\it et~al}., 2002]{KAHNG02} Kahng,A.B., Mandoiu,I.I.,
Pevzner,P.A., Reda,S. and Zelikovsky,A.Z. (2002) Border length minimization in
DNA array design. In {\it Proceedings of the Second Workshop on Algorithms in
Bioinformatics}.

\bibitem[Kahng {\it et~al}., 2003a]{KAHNG03A} Kahng,A.B., Mandoiu,I.,
Pevzner,P., Reda,S. and Zelikovsky,A. (2003a) Engineering a scalable placement
heuristic for DNA probe arrays. In {\it Proceedings of the Seventh Annual
International Conference on Computational Molecular Biology}, 148--156.

\bibitem[Kahng {\it et~al}., 2003b]{KAHNG03B} Kahng, A.B., Mandoiu,I., Reda,S.,
Xu,X. and Zelikovsky,A. (2003b), Evaluation of placement techniques for DNA
probe array layout. In {\it Proceedings of the IEEE/ACM International Conference
on Computer-Aided Design}, 262--269.

\bibitem[Koopmans and Beckmann, 1957]{KOOPMANS57} Koopmans,T.C. and
Beckmann,M.J. (1957) Assignment problems and the location of economic
activities. {\it Econometrica}, {\bf 25}, 53--76.

\bibitem[Li {\it et~al}., 1994]{LI94} Li,Y., Pardalos,P.M. and Resende,M.G.C.
(1994) A greedy randomized adaptive search procedure for the quadratic
assignment problem. In Pardalos,P. and Wolkowicz,H. (eds.), {\it Quadratic
Assignment and Related Problems}, DIMACS Series in Discrete Mathematics and
Theoretical Computer Science, {\bf 16}, 237--261.

\bibitem[Oliveira {\it et~al}., 2004]{OLIVEIRA04} Oliveira,C.A.S., Pardalos,P.M.
and Resende,M.G.C. (2004) GRASP with path-relinking for the quadratic assignment
problem. In Ribeiro,C.C. and Martins,S.L. (eds.), {\it Efficient and
Experimental Algorithms}, Lecture Notes in Computer Science, {\bf 3059},
356--368, Springer-Verlag.

\bibitem[Rahmann, 2003]{RAHMANN03}
Rahmann,S. (2003) The shortest common supersequence problem in a microarray
production setting. In {\it Proceedings of the 2nd European Conference in
Computational Biology} ({ECCB} 2003), volume 19 Suppl.~2 of
{\it Bioinformatics}, pages ii156--ii161.

\end{thebibliography}

\end{document}



